% !TEX root = paper.tex
\documentclass{article}
\usepackage[margin=1in]{geometry} % Adjust document margins
\usepackage{lipsum} % Dummy text
\usepackage{multicol} % Multi-column layout for the body
\usepackage[hidelinks]{hyperref}


\title{Knowledge and Story Comprehension}
\author{Eduardo Badillo\thanks{UC Berkeley, Email: eduardo.badillo@berkeley.edu} \and Geronimo Walker\thanks{UC Berkeley, Email: 
geronimowalker@ischool.berkeley.edu} \and   Svein Gonzalez\thanks{UC Berkeley, Email: 
sggonzal@ischool.berkeley.edu}}
\date{\today}

\begin{document}

\maketitle

\begin{abstract}
\noindent % Align the abstract with the document's margin
It has been demonstrated that pre-trained language models have a good performance on numeric, and common-sense reasoning tasks with little or no fine-tunning needed. This performance is further enhanced when more context or knowledge of the input is provided to the model at the moment of inference; for it enables the model to leverage the knowledge it already contains from its pretraining and use it more effectively on downstream tasks. In this paper we study the impact of incorporating additional knowledge, with different structures and retrieval techniques, on tasks that require both reasoning and comprehension (ClozeStory Dataset) to determine a correct ending of a story. Our code is available at \href{https://github.com/sveinerss/w266_project}{Github}.
\end{abstract}

\begin{multicols}{2}
\section{Introduction}
There are two main ways in which task performance can be enhanced in pre-trained language models, by incorporating knowledge and by fine-tunning. The latter has shown that larger and larger models with fine-tunning diminish the need to integrate auxiliary knowledge (Khashabi et al., 2020; Lourie et al., 2021). Yet, incorporating knowledge on top of large-scale models (Liu et al. 2022) has shown a marked improvement on performance when no fine-tunning is involved. Suggesting that the models are able to exploit more effectively their pre-training knowledge when additional task-relevant information is provided. 

Obtaining and encoding relevant, contextual information isn't trivial. There might not be enough quality data for the task, and the way it is retrieved from the external knowledge structure and paired with the relevant keywords or entities in each sample must be also be thought of. 
The methods we will be using are knowledge graphs (Aglionbi et al. 2022) and knowledge prompting (Liu et al. 2022). We will add the relevant knowledge contained in each on top of pre-trained models, without fine-tunning to test whether their inclusion amounts to an increased performance on the story comprehension task. These approaches have been used on common-sense, numeric reasoning tasks where the query or question is configured in a syllogistic manner. But its usefulness on comprehension datasets, where samples have no straightforward outcomes, has yet to be tested. As baseline we will be using the models without knowledge structures, and standalone models (models without pre-training). 

\section{Prompt-based Knowledge}
\lipsum[3-4]

\section{Triples}
Triples are dependency representations of sentences. For example, take the sentence “ INSERT SENTENCE HERE“. Using Spacy, we can generate triples, 
most commonly Subject DASH Verb DASH Object (SVOs), to represent the sentence with the most relevant context and information. Drawing inspiration from 
INSERT HERE, where graph representations of possible answers to questions are used to determine the highest ranking answer, our model will use 
extracted triples from the sentences comprising a story and compute a cosine similarity between the triples and two possible endings. Extracted 
triples are initially captured by a simple approach, speech tagging identification. Using SPACY and the *INSERT HERE UNDERLYING TRAINING*, we 
parse through all our sentences capturing words that are labeled *INSERT HERE EXACT* into our SVO triples. As noted in *INSERT HERE*, certain 
words such as *INSERT HERE* are stronger central nodes than others since they share more dependencies.  This will be important to improve upon 
to optimally capture significant SVOs.

First, model generated to predict the proper ending uses inference. By generating four triples, one for each story sentence, we pass through the 
a bert model to generate pooled outputs. We then compute pooled outputs of each full ending and compute cosine similarity for each. This first 
attempt didnAPOSTt result in remarkable roasting, only presenting a 50 percent accuracy DASH no better than a coinflip.

However, the next models will investigate triples with better selection and ordering of the triples. Currently, all triples are weighed evenly 
and cosines are averaged *INSERT IMAGE HERE*. The next model iteration will note the ending triples of the story more using a BertSeqtoSeq model. 
% Continue with the rest of your document
\end{multicols}

\end{document}

